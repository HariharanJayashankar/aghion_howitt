\documentclass[a4paper 12pt]{article}
\usepackage[hmargin=1.2in,vmargin=1in]{geometry}
%\usepackage[latin1]{inputenc}
\usepackage{enumerate}
\usepackage{multirow}
\usepackage{url}
\usepackage{array}
\usepackage{bbm}
\usepackage{setspace}
%\usepackage{soul}
%\usepackage[T1]{fontenc}
\usepackage{color}
\usepackage[colorlinks=false,urlcolor=blue]{hyperref}
\usepackage{amsfonts} % Worcester's
\usepackage{appendix} %Added Jan 2012
\usepackage{graphicx,caption}
\usepackage{subcaption}
%\usepackage{authblk}
%\usepackage{setspace}
\usepackage[authoryear,round]{natbib}
\usepackage{endnotes}
\usepackage{amsmath, amssymb,epstopdf,sgame,booktabs}
%\usepackage{mathtools}
\usepackage{ntheorem}
%\usepackage{amssymb,mathrsfs,multirow,sgame,lscape,subfig,enumitem,float,array,booktabs}
\usepackage{amssymb,mathrsfs,multirow,sgame,pdflscape,float,array,booktabs}
%\usepackage[margin=0.75in]{geometry}
%\usepackage[hmargin=0.9in,vmargin=0.75in]{geometry}
%\usepackage{floatrow}
%\usepackage[toc,page]{appendix}
\usepackage{listings}
%\usepackage{authblk}
\usepackage{color} %red, green, blue, yellow, cyan, magenta, black, white
\definecolor{mygreen}{RGB}{28,172,0} % color values Red, Green, Blue
\definecolor{mylilas}{RGB}{170,55,241}
\usepackage{float}
\title{Effect of Tax Cuts on Growth Rates}

\begin{document}

\maketitle

\section{No Labour}

This section derives the model of endogenous growth where there is no labour. 

\subsection{Equilibrium Conditions}

We have 4 equilibrium conditions

\begin{align}
    \frac{1+g_{t+1}}{c_t} =& \frac{1}{c_{t+1}} \beta (r_{t+1}(1 - \tau_k) + 1 - \delta) \label{eq:euler} \\
    y_t =& c_t + k_{t+1} - k_t(1 - \delta) \label{eq:goods_mkt} \\
    r_t =& (\frac{k_t}{A_t})^{\alpha - 1}\alpha^2 \label{eq:capital_mkt} \\
    y_t =& A^{1 - \alpha}k_t^\alpha \label{eq:prod_func} \\
    g_t =& (\gamma - 1)\lambda(\sigma \lambda (\alpha - 1) \left(\frac{k_t}{A_t}\right)^\alpha)^{\frac{\sigma}{1 - \sigma}} \label{eq:growth_rate}
\end{align}

\ref{eq:euler} is the household euler equation, \ref{eq:goods_mkt} is the goods market clearing equation, \ref{eq:capital_mkt} is the capital market clearing equation, \ref{eq:prod_func} is the aggregate production function and \ref{eq:growth_rate} is the equilibrium growth rate. Since \ref{eq:growth_rate} is determined directly by capital stock $k_t$ we would expect that a reduction in $\tau_k$ will increase capital stock in the economy and hence increase the growth rate.

\subsection{Model Results} 
In this model, a tax cut on capital unambiguously increases growth rate/innovation rate. The equation which determines this is
"no arbitrage" condition on innovation expenditure which has the capital tax rate directly featured in it.

\begin{center}
\includegraphics[scale = 0.8]{../product/irfs_nolab.pdf}
\end{center}

\end{document}
